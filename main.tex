\documentclass{book}

\input{../shared/shared.tex}

\usetikzlibrary{positioning, shapes.geometric}

\begin{document}
\frontmatter
\title{土木工程施工基本原理提纲}
\author{H.}
\maketitle
\tableofcontents
\chapter{说明}
\par 根据《土木工程施工基本原理》第2版整理。脚注表明非必要的补充说明。
\par 本提纲作概览和查漏补缺用,复习时宜以考点为准。
\mainmatter
\chapter{土方工程}
\section{概述}
\subsection{土的工程分类}
\par 八类土,按开挖难易程度分类,劳动定额依据。

\begin{tblr}[
    caption={土的工程分类}
    ]{
    rowspec={X[c,m]X[c,m]X[c,m]X[c,m]X[c,m]X[c,m]X[c,m]X[c,m]}
    }
    松软土 & 普通土 & 坚土 & 砾砂坚土 & 软石 & 次坚石 & 坚石 & 特坚石
\end{tblr}

\subsection{土的工程性质}
\par 最初可松性(开挖后松散状态)和最后可松性(回填压实状态),其度量为该状态下的体积与初始状态体积之比。
\par 原状土经机械压实后的沉陷,一般在$\SIrange{3}{30}{\cm}$之间。
\par 渗透性、密实性、抗剪强度、土压力。
\par 超孔隙水压力是外荷载引起的孔隙水压力,会随着时间逐渐消散,有效应力增加,土体压缩变形,这个过程就是固结。
\section{场地设计标高的确定}
\subsection{场地设计标高确定的一般方法}
\par 对于平缓且无特殊要求,场地划分为若干共$n$个方格,将角点的原地形标高(等高线插入法或实地测量)标在图上。
\par 对于角点$i$,若其被$P_i$个角点共用,标高为$z_i$,则
$$
    z_0  = \frac{1}{4n} \script{\sum}[i]{} \script{P}{i} \script{z}{i}
$$
\par 根据泄水要求得到泄水坡度$i_x$及$i_y$,则任意点设计标高
$$
    \script{z}[\prime]{i} = z_0 + l_xi_x + l_yi_y
$$
\par 各角点施工高度
$$
    H_i = \script{z}[\prime]{i} - z_i
$$
\subsection{用最小二乘法原理求最佳设计平面}
\par 即最小土方量。地形复杂时将场地划分成多个,分别计算最佳设计单平面再修正。土方施工高度平方和为
\begin{align*}
    \sigma & = \sum^n P_n H_n^2                  \\
           & = \sum^n P_n(c+x_ni_x+y_ni_y-z_n)^2
\end{align*}
\par 得(方括号指对表达式在所有角点上求和)
\begin{align*}
    [P]c + [Px]i_x + [Py]i_y - [Pz]     & = 0 \\
    [Px]c + [Pxx]i_x + [Pyx]i_y - [Pzx] & = 0 \\
    [Py]c + [Pxy]i_x + [Pyy]i_y - [Pzy] & = 0
\end{align*}
\par 更精简一点
\begin{align*}
    [PQ]c + [PQx]i_x + [PQy]i_y - [PQz] = 0                          & ,\ Q \in \{1, x, y\}       \\
    \sum^i ((P_iQ_ic + P_iQ_i)x_ii_x + P_iQ_iy_ii_y - P_iQ_iz_i) = 0 & ,\ Q_i \in \{1, x_i, y_i\}
\end{align*}
\subsection{设计标高的调整}
\par 最终可松性,提高设计标高。
\par 工程余土或工程用土,提高或降低设计标高。
\par 如采用场外取土或弃土,则应相应提高或降低设计标高。
\section{土方工程量的计算和调配}
\subsection{土方工程量计算}
\par 基坑、基槽和路堤可根据其形状划分成若干拟柱,进而计算体积,$F_0$、$F_1$和$F_2$分别是中、上、下截面的面积。
$$
    V = \frac{H}{6}(F_1+4F_0+F_2)
$$
\par 这是因为
\begin{align*}
    \int_a^b x^2 & = \frac{b^3-a^3}{3}                          \\
                 & = \frac{b-a}{6} (2b^2+2ab+2a^2)              \\
                 & = \frac{b-a}{6} [b^2+4(\frac{a+b}{2})^2+a^2]
\end{align*}
\par 场地设计标高确定后,先通过在分别为一挖一填的两个相邻角点的连线上\footnote{插入法,根据两点分别的施工高度绘制一对三角形。}确定零点,再连接得到零线,即挖方区和填方区的交线;接着进行土方量的计算。
\par 四方棱柱体法
$$
    V_p = \frac{a^2}{4} \frac{(\sum H_p)^2}{\sum H}
$$
\par 三角楔柱体法需先把方格网沿地形等高线(减小绝对值)划分为三角形。公式略去。
\subsection{土方调配}
\par 目的是运输量最小或成本最低,常用表上作业法。
\par 划分调配区时注意与工程结构的平面位置、开工顺序、施工顺序相协调,大小满足土方施工主导机械的技术要求,范围和计算土方量时的方格网协调。土方运距较大或场地内土方不平衡时,可根据地形,就近取土或弃土,并将区域加入到调配区中。
\par 调配区的平均运距和施工单价视运土工具而定,铲运机或推土机平土时平均运距为土方重心间距离。
\par 运用表上作业法\footnote{详见最小费用加权流。}时,
\begin{enumerate}
    \item 挖填区分别列入表中行列,表中方格包含土方量、价格系数、假想价格系数三个量\footnote{行、列土方量之和的最大值为对应调配区的土方量。}。
    \item 先构造初始调配方案(一般使用通过最小元素法),即循环多次,每次将对应价格系数最小的土方量设为当前仍容许的最大值,直至完成调配要求。
    \item 有调配土方的方格的假想价格系数即价格系数,无调配土方的方格的假想价格系数根据任意矩形\footnote{沿水平、竖直方向。}的四个方格内对角线上假想价格系数之和相等\footnote{所以假想价格系数其实是当前方案下通过调整从别的方格调配出入当前方格的价格系数。}。即
          $$
              c_{ef}^{\prime} + c_{pq}^{\prime} = c_{eq}^{\prime} + c_{pf}^{\prime}
          $$
          校验数$\lambda_{ij} = c_{ij} - c_{ij}^{\prime}$。若校验数均为正,则方案最优,否则选择一个校验数为负\footnote{表明从其它方格调整来当前方块优于当前方案}的方格,构造包含它的仅在有调配数字的方格转弯的闭回路\footnote{同样,只有水平或竖直方向的边。},然后取奇数次转角上的调配量中最小的数字作为调整量,然后对奇数次转角上的调配量减去该数字,偶数次转角上的调配量加上该数字。重复校验调整步骤直至最优。
\end{enumerate}
\section{土方工程的准备与辅助工作}
\subsection{土方工程施工前的准备工作}
\begin{enumerate}
    \item 清理场地内地面及地下各种障碍物,包括旧房、古墓、电讯设备、管线、建筑、树木、耕植土和淤泥。
    \item 排除地面水和雨水。
    \item 修筑道路和水电等临时设施。
    \item 材料、机具和土方机械的进场。
    \item 土方工程测量放线。
    \item 根据设计做好辅助工作,如边坡稳定、基坑基槽支护、降低地下水等。
\end{enumerate}
\subsection{土方边坡及其稳定}
\par 土方边坡坡度为高宽比,其倒数坡度系数为宽高比。应考虑土质、开挖深度、施工工期、地下水水位、坡顶荷载及气候条件因素。如地下水水位低于基底且开挖土层湿度正常,短期内可挖成直壁不加支撑。
\par 引起下滑力增加的因素有
\begin{enumerate}
    \item 坡顶上堆物、行车等荷载;
    \item 雨水或地面水渗入土中使含水量提高,自重增加;
    \item 地下水渗流产生动水压力;
    \item 土体竖向裂缝中的积水产生侧向静水压力等。
\end{enumerate}
\par 引起抗剪强度降低的因素有
\begin{enumerate}
    \item 气候使土质松软;
    \item 土体含水量增加产生润滑;
    \item 饱和的细沙、粉沙受振动而液化。
\end{enumerate}
\par 因此,要注意及时排除雨水、地面水,防止坡顶集中荷载及振动,必要时可采用钢丝网细石混凝土(或砂浆)护坡面层加固。永久性土方边坡则应做好永久性措施。
\subsection{土壁支护}
\par 建筑稠密或有地下水渗入时多不可能按要求的坡度放坡开挖,这就需要基坑基槽支护。其主要作用是支撑土壁,钢板桩、混凝土板桩和水泥土搅拌桩等还兼有不同程度隔水作用。根据受力状态可分为
\begin{enumerate}
    \item 横撑式支撑\footnote{顾名思义,两边挡土板通过其间的横撑对撑。}。
          开挖较窄的沟槽时多用。
          \begin{itemize}
              \item 水平挡土板式\footnote{横撑固定着的立柱支撑着横板作用于土壁上。}。
                    \begin{itemize}
                        \item 间断式。湿度小的黏性土,挖土深度小于$\SI{3}{\m}$。
                        \item 连续式。松散、湿度大的土。
                    \end{itemize}
              \item 垂直挡土板式\footnote{横撑支撑着竖板作用于土壁上。}。松散和湿度很高的土,挖土深度不限。
          \end{itemize}
          土压力的分布于土的性质、土坡高度、支撑的形式和变形有关,通常分密砂、松砂和黏土几种简化图形进行计算。
    \item 土钉墙\footnote{在土体中设钢筋或钢管(土钉,一般斜向下)并结合坡面的钢筋网喷射混凝土面板。}。主动嵌固作用。土钉群与土体共同作用,制约土体变形,并与之构成整体(复合土体),大大提高原土体强度和刚度,还能改变变形与破坏形态。\footnote{可以设想土体有沿滑动面滑动的趋势时,土钉群和面板的反作用力。}可以与止水帷幕、微型桩等复合。
          \par 适用于地下水位上或降水后的人工填土、粘性土和弱胶结沙土,深度不大于$\SI{5}{\m}$,不宜用于含水量丰富的细沙、淤泥质土和砂砾卵石层以及没有自稳能力的淤泥和饱和软弱土层。\footnote{土钉墙需要土钉与土体之间有一定黏结强度,并且是让复合土体共同承担自重和荷载而没有外加支撑。}
          \par 计算土钉参数时,假定破裂面是与水平面成$\frac{\beta+\varphi_{ak}}{2}$的平面\footnote{从坡脚出发向土体内延伸。取平面应为简化故。}。假定破裂面外的土体分层计算。
          \begin{itemize}
              \item 计算土钉长度时按照假定破裂面外总黏结强度不小于轴向荷载标准值与安全系数$1.4$之积的原理计算,轴向荷载标准值与土钉倾角的余弦值之积为相应范围内土体的侧压力与坡面倾斜的荷载折减系数之积。
              \item 计算土钉杆体截面面积时按照杆体截面抗拉强度不小于土钉验收抗拔力与安全系数$1.15$之积的原理计算,验收抗拔力为假定破裂面外总黏结强度与土钉的工作系数(取$0.8\sim1.0$)。
          \end{itemize}
          \par 整体稳定性分析可采用简化圆弧滑移面条分法。按照土、土钉、预应力锚杆、截水帷幕及微型桩\footnote{通常来说抗滑作用由大到小。}各自的整体稳定性分项抗力系数(产生的抗滑力矩与土体下滑力矩之比)与相应的组合作用折减系数(土没有该系数,乘以1即可)之积的和不小于整体稳定性安全系数的原理计算。
          \par 还应验算坑底抗隆起稳定性,有透水可能的还应验算抗渗流稳定性,坑底存在承压水时还应验算抗突涌稳定性。
          \begin{figure*}[htbp]
              \caption{土钉墙施工工艺流程}
              \begin{center}
                  \begin{tikzpicture}[node distance=10pt]
                      \node[draw] (step 1)  {开挖工作面};
                      \node[draw, right=of step 1] (step 2)  {修正边坡};
                      \node[draw, right=of step 2] (step 3)  {喷射第一层混凝土};
                      \node[draw, right=of step 3] (step 4)  {钻孔};
                      \node[draw, right=of step 4] (step 5)  {安设土钉};
                      \node[draw, below=of step 5] (step 6)  {注浆};
                      \node[draw, left =of step 6] (step 7)  {绑扎钢筋网};
                      \node[draw, left =of step 7] (step 8)  {喷射第二层混凝土};
                      \node[draw, left =of step 8] (step 9)  {下一个工作面施工};

                      \draw[->] (step 1) -- (step 2);
                      \draw[->] (step 2) -- (step 3);
                      \draw[->] (step 3) -- (step 4);
                      \draw[->] (step 4) -- (step 5);
                      \draw[->] (step 5) -- (step 6);
                      \draw[->] (step 6) -- (step 7);
                      \draw[->] (step 7) -- (step 8);
                      \draw[->] (step 8) -- (step 9);
                  \end{tikzpicture}
              \end{center}
              \caption{复合土钉墙施工工艺流程}
              \begin{center}
                  \begin{tikzpicture}[node distance=10pt]
                      \node[draw] (step 1)  {止水帷幕施工};
                      \node[draw, right=of step 1] (step 2)  {微型桩施工};
                      \node[draw, right=of step 2] (step 3)  {开挖工作面};
                      \node[draw, right=of step 3] (step 4)  {修整边坡};
                      \node[draw, below=of step 4] (step 5)  {土钉施工};
                      \node[draw, left =of step 5] (step 6)  {挂网、喷射混凝土施工};
                      \node[draw, left =of step 6] (step 7)  {锚杆施工};
                      \node[draw, left =of step 7] (step 8)  {下一个工作面施工};

                      \draw[->] (step 1) -- (step 2);
                      \draw[->] (step 2) -- (step 3);
                      \draw[->] (step 3) -- (step 4);
                      \draw[->] (step 4) -- (step 5);
                      \draw[->] (step 5) -- (step 6);
                      \draw[->] (step 6) -- (step 7);
                      \draw[->] (step 7) -- (step 8);
                  \end{tikzpicture}
              \end{center}
          \end{figure*}
          \par 自上而下分段分层(每层高度取决于土体直立能力,一般取土钉竖向间距以便施工)进行,机械开挖后辅以人工修正。
          \par 喷射第一层混凝土支护前,坡面需干燥平实,喷射时分段,段内自下而上,厚度不小于$\SI{40}{\mm}$,喷头保持与受喷面垂直,距离为$\SIrange{0.6}{1}{\m}$,而后铺设钢筋网,间隙不小于$\SI{20}{\mm}$。
          \par 土钉安设方法有钻孔插入法(需全长注浆,可分一次到两次)和直接打入法,上层土钉注浆体(若有注浆)及喷射混凝土面层达到设计强度的$70\%$后方可开挖下层土方。
          \par 施工完毕后进行土钉抗拉、喷射混凝土钻孔检测。
    \item 板桩式支护结构,又分为悬臂式和支护式。支护式由挡墙系统和支撑或拉锚系统(悬臂式不设)两大系统组成。悬臂式位移较大,多用于$\SIrange{3}{4}{\m}$深的浅基坑,一般广泛采用支撑式板桩。
          \par 挡墙系统常用槽钢、钢板桩(平板式和波浪式\footnote{隔水能力强,波浪式截面抗弯能力好。},通过锁口互相连接,完毕后可拔出重复使用)、钢筋混凝土板桩、灌注桩、地下连续墙和SMW工法等。
          \par 支撑系统一般采用大型钢管、H型钢、格构式钢支撑或现浇钢筋混凝土支撑。拉毛系统的材料一般用钢筋、钢索、型钢或土锚杆。根据开挖深度和挡墙系统的截面性能可设一道或多道支点。
          \par 板桩工程事故主要有五方面原因
          \begin{itemize}
              \item 板桩的入土深度不够,入土部分移动而滑坡;
              \item 支撑或拉锚的强度不够;
              \item 拉锚长度不足,锚碇失去作用;
              \item 板桩刚度不够,失稳弯曲;
              \item 板桩位移过大,周边环境破坏。
          \end{itemize}
          \par 因此,板桩的设计五大要素为入土深度、截面弯矩、支点反力、拉锚长度和板桩位移。
          \par 以单支点\footnote{支点指支撑和拉锚。}为例,根据入土深度与基坑深度比值的大小,单支点板桩变形也不同,分为自由支承和嵌固支承。\footnote{前者入土浅,整个板桩都向坑内变形,底端转动且无位移或有微小位移,桩前土发挥被动土压力。后者底端有一段既无位移也无转角。}
          \par 对于单支点嵌固板桩,常用简化计算方法---相当梁法。将土压力简化为线性分布,则板桩有反弯点\footnote{反弯点上方的部分即相当梁。},简化取反弯点在土压力零点\footnote{桩前后被动、主动土压力在该点相等。该简化可避免求支承力。由于线性,相当梁上桩前后土的合力也相等,反弯点剪力与支承力平衡。},再根据桩前后土压力求出反弯点下另一个弯矩零点。两零点之间的长度需要乘以$1.2$以修正,与反弯点入土深度相加得到板桩入土深度;在两零点之间剪力为零处\footnote{由于线性,为中点。}求得最大弯矩。
          \begin{align*}
              \script{h}{c1} & = \frac{\script{K}{a}h}{\script{K}{p}-\script{K}{a}}                        \\
              \script{h}{0}  & = \sqrt{\frac{6\script{R}[\prime]{c}}{\gamma(\script{K}{p}-\script{K}{a})}} \\
              \script{h}{d}  & = \script{h}{c1} + 1.2 \script{h}{0}
          \end{align*}
          为解出\script{R}[\prime]{c},对上部板桩以拉锚力作用点应用力矩平衡方程
          \begin{align*}
              \script{R}{c} (\script{h}{c1}+\script{h}{T1}) + \gamma\script{K}{p}\script{h}{c1} (\script{h}{T1}+\frac{2\script{h}{c1}}{3}) & = \gamma\script{K}{a} (h+\script{h}{c1}) (\script{h}{T1}+\script{h}{c1}-\frac{\script{h}{T1}+\script{h}{c1}}{3})                                                                          \\
              \script{R}{c}                                                                                                                & = \frac{\gamma\script{K}{a} (h+\script{h}{c1}) (2\script{h}{T1}+3\script{h}{c1}-h)-\gamma\script{K}{p}\script{h}{c1} (3\script{h}{T1}+2\script{h}{c1})}{3(\script{h}{c1}+\script{h}{T1})}
          \end{align*}
          \par 而$\script{R}[\prime]{c}=\script{R}{c}$,即可解出。
          \par 拉锚长度应保证锚碇或锚座板在静止土楔滑移线之外,它引起的被动土楔滑移线和板桩位移引起的主动土楔滑移线不能相交。\footnote{主动、被动土楔滑移线与受力(主动为重力)方向成$(90\degree-\varphi)/2$角斜向外,静止土楔滑移线与受力方向成$90\degree-\varphi$角斜向外。注意计算嵌固板桩时,板桩位移引起的被动土楔滑移线以反弯点为起点。}
          \par 围檩\footnote{支护板桩上部的横梁,起连接固定作用。}可用型钢或现浇混凝土结构,前者按简支梁计算,后者按连续梁计算。
          \par 一般钢板桩、混凝土板桩采用打入法。灌注桩和地下连续墙采用就地成孔(槽)现浇的方法。板桩施工要正确选择打桩方法、打桩机械和流水段划分,以保证刚度和防水性,且墙面要平直,封闭式板桩墙还要封闭合拢。
          \par 钢板桩的打桩方法通常有单独打入法\footnote{一角开始逐块插打,每块不停顿,打设快,但易向一边倾斜不易纠正,墙体平直度难控制,一般长度小于$\SI{10}{\m}$。}、围檩插桩法\footnote{用围檩支架作板桩打设导向装置。分单面围檩和双面围檩。可以保证尺寸准确和钢板桩垂直,但施工速度慢。}、分段复打法\footnote{一定数目钢板桩沿单层围檩插入土中从两端开始依次打入,打好后电焊固定在围檩上,最终形成屏风墙。桩锤重量一般为钢板桩2倍,不能过宽。}。
          \par 地下工程结束后钢板桩一般都要拔出,需正确选取方法和顺序,并应注浆填充。
    \item 重力式支护结构。利用深层搅拌桩机将水泥与土一起搅拌,形成柱状的加固土(搅拌桩),用于支护结构时水泥掺量(单位体积掺入水泥质量与土的重力密度之比)通常为$\SIrange{12}{15}{\percent}$,渗透系数很小。适用于$\SIrange{4}{6}{\m}$的基坑,最大$\SIrange{7}{8}{\m}$。
          \par 设计主要包括整体稳定、抗倾覆稳定、抗滑移稳定、位移等。有时还应验算抗渗、墙体应力、地基强度等。施工质量对位移的影响不可忽略。
          \par 通常布置成格栅式。置换率(加固土面积与水泥土墙总面积之比)为$\numrange{0.6}{0.8}$。
          \par 常用深层搅拌桩机分动力头式和转盘式两大类。(多头可在一个工艺流程施工多根桩)SMW工法可以利用装有三轴搅拌钻头的钻机在地层中连续建造水泥土墙并插入芯材。
          \par 成桩工艺可采用“一次喷浆、二次搅拌”或“二次喷浆、三次搅拌”工艺(水泥掺量小、土质松时可用前者,反之可用后者)。应注意水泥浆配合比、搅拌制度、水泥浆喷射速率与提升速率的关系和每根桩的水泥浆喷注量,以保证注浆的均匀性与桩身强度;还应注意桩的垂直度、桩的搭接,以保证水泥土墙的整体性和抗渗性。
\end{enumerate}
\subsection{降水}
\par 降水方法可分为重力降水和强制降水\footnote{用水量较大、水位较大或土质为细沙、粉沙,易产生流沙、边坡塌方及管涌等时往往采用。}。
\par 集水井降水是在开挖时在坑底设置集水井,井沿周围或中央开挖排水沟,使水在重力下流入,用水泵抽出。一般设在基础范围外、地下水流上游,基坑较大时可以在基础范围内设置盲沟。根据地下水量、基坑平面形状、水泵能力,每隔$\SIrange{20}{40}{\m}$设一集水井,一般宽$\SIrange{0.6}{0.8}{\m}$,深度随挖土加深,通常低于挖土面$\SIrange{0.7}{1.0}{\m}$,井壁可用竹、木等简单加固。挖至设计标高时井底应低于坑底$\SIrange{1}{2}{\m}$,并设碎石滤水层。集水井降水比较简单经济,影响小。
\par 井点降水是在开挖前预先埋设滤水管(井),利用真空原理不断抽出地下水。作用有
\begin{itemize}
    \item 防止地下水涌入坑底,防止边坡由于地下水的渗流引起塌方,消除地下水位差导致的坑底土层压力,防止坑底的管涌;
    \item 减少板桩的横向荷载,消除地下水渗流,防止出现流沙;
    \item 使土壤固结,增加地基的承载能力。
\end{itemize}
\par 易知水对土颗粒的作用方向与渗流一致,大小为水力坡度和浮重度之积。防治流沙的方法主要有水下挖土法、冻结法、枯水期施工、抢挖法、加设支护结构和降水法等。
\par 井点降水法有轻型井点和管井两类。轻型经典设备由管路系统\footnote{从末到本为滤管、井点管、弯联管、短接头和总管等。}和抽水设备\footnote{真空泵、离心泵和水气分离器(又称集水箱)。}组成。井点系统布置应根据水文地质资料、工程要求和设备条件等确定。
\par 轻型井点布置包括平面布置(井点布置形式、总管长度、井点管数量、水泵数量及位置等)和高程布置(埋设深度)。布置和计算时先确定平面布置,再确定高程布置(确定滤管上口至总管埋设面距离),接着计算井点管数量等,最后根据问题调整设计。平面布置中
\begin{itemize}
    \item 单排布置适用于基坑(槽)宽度小于$\SI{6}{\m}$且降水深度不超过$\SI{5}{\m}$的情况。应布置在地下水上游侧,两侧延伸长度不宜小于坑(槽)宽度。
    \item 双排布置适用于基坑宽度大于$\SI{6}{\m}$或土质不良的情况。
    \item 环形布置适用于大面积基坑。
    \item $U$形布置,不封闭的一段应设在地下水的下游。
\end{itemize}
\par 井点管布置应离坑边$\SIrange{0.7}{1}{\m}$,以防塌土漏气。
\par 一般基底至降低后的地下水位线距离取$\SIrange{0.5}{1}{\m}$,降水后的水力坡度(假设恒定)为
\begin{itemize}
    \item 单排布置时$1/4\sim1/5$;
    \item 双排布置时$1/7$;
    \item 环形布置时$1/10$。
\end{itemize}
\par 实际工程中,井点管定型,通常露出地面$\SI{0.2}{\m}$。总管长度根据基坑上口尺寸或基槽长度确定,进而根据水泵负荷长度确定水泵数量。
\par 为了确定井点管数量还需计算井点系统涌水量。水井根据地下水有无压力,分为承压井和无压井,根据底部是否为不透水层,分为完整井和非完整井。
\par 计算无压完整井时,假定某一过水面(以井点为中心的圆柱面)水力坡度是恒值,根据含水层厚度、断面流量恒定和达西定律可求出水位降落曲线和流量。非完整井的含水层厚度取有效含水深度,有效含水深度根据井点管处水位降落值和井点管中水位计算查表得到,但不超过实际含水层厚度。
\par 轻型井点施工过程分为准备工作(井点设备、动力、水源、必要材料的准备、排水沟的开挖、附近建筑物的标高观测和实施防止附近建筑物沉降的措施)、井点系统的埋设(先设置总管,再设井点管,用弯联管将井点与总管接通,然后安装抽水设备。一般用水冲法进行,分为冲孔与埋管两个过程。需注意不让滤管触底、保证砂滤层填灌质量,并用粘土封口。开始抽水后一般不应停抽)、使用和拆除。
\par 轻型井点降水优点许多,但抽水影响范围较大,可达百米至数百米,且会导致周围土壤固结、地面沉陷。可采用回灌井点法消除地面沉陷,将抽取的水沉淀后注入井点外数米的管内形成水墙,此时抽水管的抽水量约增加$10\%$,可适当增加抽水井点的数量。
\par 管井井点一般有疏干井(还需降低土体含水量以达到提高边坡稳定性、增加坑内土体固结强度、便于机械挖土和提供坑内干作业施工条件等目的时)、降压井(软土地区承压水一般接近上覆土层自重应力,此时基坑开挖后需降低承压水压力以防止基坑突涌)和混合井三种。
\subsection{基坑支护工程的现场监测}
\par 由于地下工程影响因素比较复杂,实际工作状况往往与设计计算值不一致,需对基坑支护工程现场检测。设计阶段,设计人员根据具体情况,对现场检测提出观测项目、测点布置、观测精度、观测频度和临界状态报警值等具体要求,监测人员据其在基坑开挖前制定出包括检测目的与内容、测点布置、使用的仪器、检测精度、观测方法、观测周期、监测项目报警值、检测结果处理要求和监测结构反馈制度等现场检测方案。
\par 支护结构检测主要包括(标下划线的项目不可缺少)
\begin{itemize}
    \item \uline{水平位移监测}、竖向位移监测、深层水平位移监测、\uline{沉降监测}、倾斜检测;
    \item \uline{裂缝检测};
    \item 支护结构内力监测、土压力监测、孔隙水压力监测;
    \item 地下水位监测。
\end{itemize}
\par 基坑工程监测方法应根据基坑类别、设计要求、场地条件、当地经验和方法适用性等因素综合确定,以仪器(根据检测项目选择,主要有水准仪、全站仪、测斜仪、回弹仪、轴力仪、土压力计和水压力计等,需满足观测精度、量程、稳定性、可靠性的要求)观测为主,辅以巡视检查。
\par 监测的基本要求有
\begin{itemize}
    \item 严格执行;
    \item 动态及时观测;
    \item 可靠,精度符合要求;
    \item 发现异常及时预报、应急补救;
    \item 观测记录完整,有形象的图表、曲线和观测报告。
    \item 监控值以设计指标或规定为依据。
\end{itemize}
\section{土方工程的机械化施工}
\par 施工过程包括土方开挖、运输、填筑和压实。尽量采用机械化施工。主要挖土机械有
\begin{itemize}
    \item 推土机。以切土和推运土方为主,适于开挖一至三类土,多用于平整场地、开挖深度不大的基坑,移挖作填,回填土方,堆筑堤坝以及配合挖土机集中土方、修路开道等。\footnote{履带式拖拉机上安装推土板等而成,多用油压操纵。操纵灵活,运转方便,所需工作面较小,行驶速度快,易于转移,能爬$30\degree$左右的缓坡,应用范围较广。经济运距在\SI{100}{\m}以内,效率最高为\SI{60}{\m}。可采用槽型推土、下坡推土以及并列推土等方法提高效率。}
    \item 铲运机。适于开挖一至三类土,常用于坡度为\SI{20}{\degree}以内的大面积土方挖、填、平整、压实,大型基坑开挖和堤坝填筑等。\footnote{按行走方式分为自行式和拖式两种,按铲斗操纵系统可分为机械操纵和液压操纵。操纵简单,不受地形限制,能独立工作,行驶速度快,生产效率高。线路可采用环形路线或$8$字路线。适用运距为\SIrange{600}{1500}{\m},效率最高为\SIrange{200}{350}{\m}。可采用下坡铲土、跨铲法、推土机助铲法等缩短装土时间,提高土斗装土量。}
    \item 挖掘机。\footnote{按行走方式分为履带式和轮胎式,按传动方式分为机械传动和液压传动。}
          \begin{itemize}
              \item 装载机。主要用于铲装土壤、砂土、石灰、煤炭等散状物料,也可对矿石、硬土等作轻度铲挖作业。\footnote{作业速度快、效率高、机动性好、操作轻便。}
              \item 正铲挖掘机。适于开挖含水量较小的一至四类土和经爆破的岩石及冻土,挖土时\uline{前进向上,强制切土}。\footnote{生产率主要决定于每斗作业的循环延续时间,为提高生产率,除工作面高度满足装满土斗的要求外,还要考虑开挖方式、与运土机械配合。}
              \item 反铲挖掘机。适于开挖一至三类的沙土或黏土。主要用于开挖停机面以下的土方,挖土时\uline{后退向下,强制切土}。一般最大挖土深度为\SIrange{4}{6}{\m},经济合理为\SIrange{3}{5}{\m}。\footnote{需要配备运土汽车,可以采用沟端开挖法或沟侧开挖法。}
              \item 抓铲挖掘机。适用于开挖较松软的土。挖土时\uline{直上直下,自重切土}。\footnote{对施工面狭窄而深的基坑、深槽、深井采用。}还可用于挖取水中淤泥(易被吸住,谨防翻车),装卸碎石、矿渣等松散材料。
              \item 拉铲挖掘机。适用于一至三类土,可开挖停机面以下的土方,如较大基坑基槽和沟渠,挖取水下泥土,也可用于填筑路基、堤坝等,挖土时\uline{后退向下,自重切土}。但开挖的边坡及坑底平整度较差。开挖方式有沟端开挖和沟侧开挖。
          \end{itemize}
\end{itemize}
\par 选择土方机械应依据土方工程的类型及规模\footnote{开挖或填筑的断面深度及宽度、工程范围的大小、工程量的多少}、地质水文气候条件(土的类型、土的含水量、地下水)、机械设备条件(现有土方机械的种类、数量和性能)来选择。
\subsection{土方的填筑与压实}
\par 填方土料应为强度高、压缩性小、水稳定性好、便于施工的土、石料。如设计无要求,应符合
\begin{itemize}
    \item 碎石类土、沙土和爆破石渣(粒径不大于每层铺厚的$2/3$)可用于表层下的填料。
    \item 黏性土含水量需要符合压实要求,作为路基填料时还需设有良好的排水设施。
    \item 有压实要求时不得使用碎块草方和有机质含量大于$8\%$的土。
    \item 一般不用淤泥和淤泥质土,但在软土或沼泽地区,经过处理、含水量符合压实要求后可用于次要部位。
\end{itemize}
\par 严格控制含水量,施工前应校验。含水量过大时应采用翻松、晾晒、风干,或换土回填、均匀掺入干土或其他吸水材料、打石灰桩等方法,含水量偏低则预先洒水湿润以便压实。
\par 填土可采用人工填土和机械填土。从低处分层开始,逐层压实。
\par 填土的压实方法有
\begin{itemize}
    \item 碾压适用于大面积填土工程。碾压机械有压路机平碾、羊足碾和气胎碾,也可利用运土过程借助运土工具。
    \item 夯实主要用于小面积填土,可以夯实黏性土或非黏性土。优点是可以压实较厚的土层。夯实的机械有夯锤、内燃夯土机和蛙式打夯机等。
    \item 振动压实主要用于压实非黏性土,机械主要是振动压路机、平板振动器等。
\end{itemize}
\par 填土压实质量主要影响因素为
\begin{itemize}
    \item 压实功,提升压实功对重度的提高作用幅度递减。对不同的土应根据压实机械和密实度要求选择合理的压实遍数,松土应该先用轻碾,再用重碾。
    \item 土的含水量,含水量适中时摩阻力小、缝隙大,易压实。
    \item 每层的铺土厚度,铺土厚度应小于压土的有效作用深度,还应考虑最经济的土层厚度。
\end{itemize}
\par 填土压实后应达到一定的密实度和含水量要求。压实度为控制干重度与最大干重度之比。控制干重度用“环刀法”、灌砂法或灌水法测定,最大干重度则用击实试验(标准击实试验分轻型标准、重型标准两种,一般选用要求更高的后者)确定。
\chapter{桩基础工程}
\section{概述}
\par 桩基是土木工程中常用的深基础形式,它由桩和承台组成,按承载性质,可分为摩擦桩、端承摩擦桩、端承桩和摩擦端承桩,前后两对分别为摩擦型桩、端承型桩。按挤土状况分为非挤土桩、部分挤土桩和挤土桩。按施工方法分为预制桩和灌注桩。按作用可分为建筑物(构筑物)的基础和深基坑的围护结构。
\par 复合地基是天然地基中部分土体经处理得到增强或置换,或天然地基中布置加筋材料。常见形式有砂石桩法、水泥粉煤灰碎石桩法、夯实水泥土桩法、高压喷射注浆法、石灰桩法。
\par 桩身全身或者部分埋入土中,顶部和承台或地下室底板连成一体,在其上修筑上部结构。桩基础按施工方法不同可以分为预制桩(预制沉桩)和灌注桩(桩位上成孔灌注)。
\par 深基坑围护结构的桩基分为重力式围护结构(以挡墙的自重和刚度保护基坑侧壁,既挡土又挡水,常见形式有水泥土搅拌桩和土钉墙)、排桩与板墙式围护结构的桩基(有排桩或地下连续墙等作为基坑围护挡墙,有时另设内支撑或加外拉的土层锚杆。其常见的桩基形式有(钻孔)灌注桩、(钢)板桩、SMW工法、地下连续墙等)和截水帷幕的桩基(在基坑围护中设置于挡土墙外侧用于阻截或减少基坑侧壁及基坑底地下水流入基坑而采用的连续止水体。其常见的桩基形式有水泥土搅拌桩、高压旋喷桩和压密注浆等)。
\section{预制桩施工}
\subsection{预制桩的准备、起吊、运输和堆放}
\par 混凝土预制桩能承受较大的荷载、坚固耐久、桩身质量易控制、施工速度快,但施工中对周围环境影响较大。常用的是混凝土实心方桩和预应力混凝土空心管桩。预制方桩多在施工现场或者附近就地预制,较短的桩也可在预制厂生产,预应力管桩则均为工厂生产(达到设计强度后方能出厂)。必须在下层桩或邻桩的混凝土达到设计强度$30\%$以后方可进行。
\par 多节桩上下节尽量在同一中轴线上制作。
\par 桩的混凝土强度达到设计强度的$70\%$方可起吊,达到$100\%$方可运输和打桩。
\par 钢桩可采用管桩、H型或者其他异形钢材制作。其优点在于
\begin{enumerate}
    \item 重量轻、刚性好,装卸、运输、堆放方便,不易损坏。
    \item 承载力高,由于钢材强度高,能够有效地打入坚硬土层,从而获得极大的单桩承载力。
    \item 桩长易于调节,实际施工中可根据需要采用接长或切割的办法调节桩长。
    \item 排土量小,对邻近建筑物及周围地基的扰动也较小,可避免土体隆起。
    \item 接头连接简单,采用电焊焊接,操作简便,强度高,安全性好。
    \item 工程质量可靠,施工速度快。
\end{enumerate}
\par 缺点在于
\begin{enumerate}
    \item 钢材用量大,工程造价较高;
    \item 打桩机具设备复杂,振动和噪声大;
    \item 桩材易腐蚀。
\end{enumerate}
\subsection{预制桩的连接}
\par 混凝土预制方桩的常用接桩方法有焊接法、锚接法和装配式可调平刚性接头。预应力混凝土管桩拼接时宜采用端板焊接连接或者机械接头(接头连接强度不应小于管桩桩身强度)连接。钢桩焊接接头宜采用等强连接。
\subsection{预制桩的施工质量控制}
\par 沉桩的质量控制包括沉桩前、沉桩过程中的控制以及施工后的质量检查。施工前应做外观及强度检验。桩位放样允许偏差:对群桩为$\SI{20}{\mm}$,对单排桩为$\SI{10}{\mm}$。沉桩前应先进行沉桩试验。
\par 根据桩群的密集程度,可选用由一侧向单一方向进行、自中间向两个方向对称进行、自中间向四周对称进行、分段进行。先深后浅,先大后小,先长后短。施工中应采取适当措施,减小挤土效应。实际施工中常用的措施有
\begin{itemize}
    \item 预钻孔沉桩法;
    \item 设置隔离板桩;
    \item 挖沟防振(应力释放沟);
    \item 砂井排水(或塑料排水板排水);
    \item 限制沉桩速率。
\end{itemize}
\par 采用锤击法时,应根据地质条件、桩型、桩的、密集程度、单桩竖向承载力及现有施工条件或相关规范选用桩锤,遵循“\uline{重锤轻击、低提重打}”。打桩设备包括桩锤(主要有落锤、蒸汽锤、柴油锤和液压锤)、桩架(支撑桩身和悬吊桩锤,有轨道行驶的多能桩架、装在履带底盘上的履带式桩架等基本形式)和动力装置。
\par 采用静压法时,过程中不得调整和校正垂直度,穿越硬土层或进入持力层的过程中不得停止施工。打桩完成后应对桩基承载力和桩身质量进行检查,有静载荷试验法、动测法和Osterberg测桩法。桩身质量检验常用方法有低应变反射波法、声波透射法和钻芯法。主要优点为\begin{itemize}
    \item 低噪声、无振动、无污染、场地整洁、施工文明程度高,适合城市施工;
    \item 施工速度很快,可以$\SI{24}{\hour}$连续施工,缩短建设工期,创造时间效益,从而降低工程造价;
    \item 桩定位准确,不易产生偏心,可提高桩基施工质量;
    \item 桩身不受锤击应力,桩身混凝土强度可降低,配筋可减少,可降低工程造价。
\end{itemize}
\par 主要缺点为\begin{itemize}
    \item 具有一定挤土效应,对周围建筑环境及地下管线有一定的影响,边桩中心到相邻建筑物的间距要求较大;
    \item 施工场地的地耐力要求较高,在新填土、淤泥土及积水浸泡过的场地施工易陷机,对表土层软弱的地方需事先进行处理;
    \item 过大的压桩力(夹持力)易将管桩桩身夹破夹碎,是管桩出现纵向裂缝;
    \item 在地下障碍物或孤石较多的场地施工,容易出现斜桩甚至断桩。
\end{itemize}
\par 常用的静力压桩机有机械式和液压式。
\par 其它沉桩方法还有振动法和水冲法。
\section{灌注桩施工}
\subsection{灌注桩施工与质量控制}
\par 根据成孔工艺不同,可分为钻孔灌注桩、冲孔灌注桩、套管成孔灌注桩、爆扩成孔灌注桩、现夯扩沉管灌注桩和钻孔压浆灌注桩。以上仅钻孔灌注桩为非挤土桩;其余应间隔成孔、避免相邻孔初凝后终凝前成孔,五桩以上群桩成孔先中间后四周,临近爆扩桩可根据桩距采用单爆或联爆法。
\par 摩擦桩应以设计桩长控制成孔深度,端承摩擦桩必须保证设计桩长和桩端进入持力层深度。锤击沉管法成孔时应以成孔为主时,桩管入土深度控制应以标高为主,贯入度控制为辅。
\par 端承型桩,采用钻、冲、挖成孔时,必须保证桩端进入持力层的设计深度。锤击沉管法成孔时,桩管入土深度控制应以贯入度为主,控制标高为辅。
\par 制作钢筋笼时应对钢筋规格、焊条规格、品种、焊口规格、焊缝长度、焊缝外观及质量、主筋和箍筋的制作偏差等进行检查。分段接头宜采用焊接或机械式连接(钢筋直径大于$\SI{20}{\mm}$);除非施工工艺有特殊要求,加劲箍宜设置在主筋外侧。
\par 浇筑混凝土前对已成孔的中心位置、孔深、孔径、垂直度、孔底沉渣厚度(直接影响桩的承载力及其沉降量)等进行认真检查。灌注桩施工后也应进行桩的承载力检测与桩身质量检查。
\subsection{钻孔灌注桩成桩}
\par 根据地下水位和图纸情况,分为干作业成孔灌注桩和湿作业成孔灌注桩(泥浆护壁钻孔灌注桩)。
\par 干作业成孔灌注桩施工工艺流程为场地整平、测定桩位、钻孔(螺旋钻孔、洛阳铲成孔、人工挖孔)、清孔、下钢筋笼和浇筑混凝土。
\par 泥浆护壁钻孔灌注桩施工工艺流程为测定桩位、埋设护筒、桩机就位、制备泥浆、成孔、清孔、下钢筋笼和水下浇筑混凝土。
\par 护壁泥浆制备的方法根据土质情况确定,黏土中可在孔中注入清水自造泥浆护壁,其它土层应注入制备泥浆。护壁泥浆由黏性土或膨润土和水拌合而成,还可掺入加重剂、分散剂、增黏剂和堵漏剂等掺合剂。
\par 泥浆循环方式分为正循环(泥浆由钻孔流出由孔口将土渣带出)和反循环(泥浆由钻孔与孔壁间的环状间隙流入钻孔,由钻孔吸入),根据桩型、钻孔深度、土层情况、泥浆排放条件、允许沉渣厚度等进行选择,对孔深较大的端承型桩和粗粒土层中摩擦型桩,宜采用反循环成孔及清孔,也可根据土层情况采用正循环钻进、反循环清孔。
\par 成孔机械有回旋钻机、潜水钻机、冲击钻等。
\par 清孔时应不断置换泥浆。
\subsection{沉管灌注桩成桩}
\par 采用套管成孔,套管是采用锤击打桩法或振动打桩法。
\par 利用锤击沉桩设备沉管、拔管,称为锤击沉管灌注桩(宜用于一般黏性土、淤泥质土、沙土和人工填土地基,不宜在密实的中粗砂、砂砾石、漂石层中使用);利用激振器振动沉管、拔管称为振动沉管灌注桩(单打法适用于含水量较大的土层,并宜采用预制桩尖,反插法及复打法适用于饱和土层),也可采用振动锤击双作用的方法沉管。
\par 沉管灌注桩适用于黏性土、粉土、淤泥质土、沙土及填土,在厚度较大、灵敏度较高的淤泥和流塑状态的黏性土等软弱土层中采用时,应制定质量保证措施,并经工艺试验成功后方可实施。
\par 沉管灌注桩施工时易发生断桩、缩颈、桩靴进水或进泥及吊脚桩等问题,施工中应加强检查并及时处理。
\subsection{爆扩成孔灌注桩}
\par 先用钻机成孔或人工挖孔,然后在孔底放入炸药,再灌入适量的混凝土,然后引爆,使孔底形成扩大头,再清孔后放入钢筋笼,浇筑混凝土成桩。能显著提高桩的承载能力,成孔简单、节省劳力和成本低廉,但施工质量要求严格,且检查质量不便。
\section{水泥搅拌桩}
\subsection{施工与质量控制}
\par 水泥搅拌桩是利用搅拌桩机将水泥喷入土体并充分搅拌,经过水泥与土的一系列物理化学反应,使软土硬结成的具有整体性、水稳定性、一定强度的地基,是软土地基处理的一种有效形式,也可用此法构建重力式支护结构;适用于正常固结的淤泥、淤泥质土、饱和黄土、泥炭土和粉土土质。水泥是固化剂的主剂。
\par 水泥搅拌桩按材料喷射状态可分为湿法(深层搅拌桩,以水泥浆为主,搅拌均匀,便于复搅,水泥土硬化时间较长,深度不宜大于\SI{20}{\m})和干法(粉法喷搅法,以水泥干粉为主,水泥土硬化时间较短,能提高桩间的强度,但搅拌均匀性欠佳,很难全程复搅,深度不宜大于\SI{15}{\m})两种,直径不应小于\SI{500}{\mm}。
\par 目前搅拌桩可布置为桩状(堤防上地基加固)、壁状(防渗加固)和块状三种形式。
\subsection{深层搅拌法}
\par 深层搅拌桩机由深层搅拌机(主机)、机架及灰浆搅拌机、灰浆泵等配套设施组成。深层搅拌桩机常用的机架有塔架式、桅杆式及履带式等三种形式。
\par 深层搅拌法成桩工艺可采用“一次喷浆、二次搅拌”或“二次喷浆、三次搅拌”工艺,水泥产量较小、土质较松可用前者\footnote{似曾相识?},反之可用后者。
\subsection{粉体喷搅法}
\par 施工时用钻头在桩位搅拌后将水泥干粉用压缩空气输入到软土中,强行拌合,使其充分吸收地下水并与地基土发生理化反应,从而形成柱状体。桩径一般为\SIlist{500;600;700}{\mm}。
\section{地下连续墙}
\par 优点是墙体刚度大,抗弯强度高,变形小,具有良好的抗渗性能,能抵抗较高的水压力;施工适应性强,可以用于各种土质条件,施工时噪音低、无振动、不挤土,可在建筑物(构筑物)密集区域施工,对邻近建筑物和地下管线的影响较小;可用于逆作法施工,即将地下连续墙与逆作法结合,形成一种深基坑和多层地下室施工的有效方法。缺点是施工工艺复杂,需要较多的专用设备,成本较高;施工中废泥浆需妥善处理,否则易污染环境。
\par 先沿设计轴线施工导墙\footnote{临时结构,作用是挖槽导向、承受挖槽机械的荷载、防止槽段上口塌方、存蓄泥浆、保证地下连续墙设计的几何尺寸和形状、并作为安装钢筋骨架的基准。},接着挖槽、清底\footnote{沉淀法和置换法}后立即在泥浆护壁的保护下进行混凝土浇筑(导管法)。地下连续墙单元槽段之间在垂直面上存在接头,常见形式有锁口管(钢管,施工后拔出)接头、锁头箱接头、隔板式接头。
\chapter{钢筋混凝土结构工程}
\section{钢筋工程}
\par 常用的钢材有钢筋(一般加工过程有调直、剪切、镦头、弯曲、连接、质量检查)、钢丝(光圆钢丝、螺旋肋钢丝和刻痕钢丝三类)和钢绞线(主要用于预应力混凝土结构中,一般有$2$根、$3$根、$7$根冷拉光圆钢丝或刻痕钢丝捻制而成)三类。
\subsection{钢筋连接}
\par 常用的连接方式为焊接连接(压焊和熔焊,前者包括闪光对焊\footnote{连续闪光焊、预热闪光焊和闪光--预热--闪光焊。}、电阻点焊\footnote{主要用于小直径钢筋的交叉连接。钢筋交叉点焊时热量集中在一点。}和气压焊\footnote{利用乙炔--氧混合气体燃烧的高温火焰对已有初始压力的两根钢筋端面接合处加热。属于热压焊。气压焊接设备主要包括加热系统与加压系统两部分。},后者包括电弧焊\footnote{钢筋电弧焊的接头形式有搭接焊接头(单面焊缝或双面焊缝)、帮条焊接头(单面焊缝或双面焊缝)、剖口焊接头(平焊或立焊)和熔槽帮条焊接头。}和电渣压力焊\footnote{先除锈,再在下部钢筋上夹牢夹具、在活动电极中扶直夹牢上部钢筋,接着装上药盒、焊药、接通电路,稳定一定时间,用手柄缓缓送下上部钢筋,稳弧规定时间后,加压顶锻。})、机械连接(挤压连接和螺纹套筒连接)和绑扎连接。钢筋与预埋件$T$形接头的焊接应采用埋弧压力焊,也可用电弧焊或穿孔塞焊。
\subsection{钢筋代换}
\par 施工中有时需要对设计图的钢筋品种和规格进行变更。代换原则主要有等强度代换(强度控制)和等面积(配筋率控制)代换。
\subsection{成型钢筋的工厂化生产}
\par 成型钢筋的工厂化生产分为原材选择和加工(核对,编制钢筋配料单,调直,切断,连接)过程。
\section{模板工程}
\subsection{模板形式}
\begin{itemize}
    \item 木模板。
          \begin{itemize}
              \item 基础模板。
              \item 柱子模板。
              \item 梁、楼板模板。
          \end{itemize}
    \item 组合模板。
          \begin{itemize}
              \item 板块与角膜。
              \item 支承件。
          \end{itemize}
    \item 大模板。
    \item 滑升模板。
    \item 爬升模板。
    \item 其他模板。台模、隧道模、永久性模板。
\end{itemize}
\subsection{模板设计}
\begin{itemize}
    \item 永久荷载。对结构不利时,可变荷载控制的组合,分项系数为$1.2$,永久荷载效应控制的组合,分项系数为$1.35$;有利时分项系数为$1$;倾覆、滑移验算时分项系数为$0.9$。
          \begin{itemize}
              \item 模板及支架自重。
              \item 新浇筑混凝土自重。普通混凝土为$\SI{24}{\kN/\m^3}$。
              \item 钢筋自重标准值。一般梁板结构中按混凝土体积计,楼板$\SI{1.1}{\kN/\m^3}$,梁$\SI{1.5}{\kN/\m^3}$。
              \item 施工人员及设备荷载标准值。
                    \begin{itemize}
                        \item 模板及支撑模板的小肋,取$\SI{2.5}{\kN^2}$均布活荷载和$\SI{2.5}{\kN/\m}$跨中集中荷载中较大弯矩值。
                        \item 支撑小肋的构件,均布活荷载$\SI{1.5}{\kN/\m^2}$。
                        \item 支架立柱及其他支撑结构构件,均布活荷载$\SI{1.0}{\kN/\m^2}$。
                    \end{itemize}
          \end{itemize}
    \item 可变荷载。分项系数一般取$1.4$,标准值大于$\SI{4}{\kN/\m^2}$则取$1.3$。
          \begin{itemize}
              \item 振捣混凝土,水平面模板$\SI{2.0}{\kN/\m^2}$,垂直面模板$\SI{4.0}{\kN/\m^2}$;
              \item 新浇筑混凝土对模板侧面的压力。取两式中的较小值;
              \item 倾倒混凝土时对垂直面模板产生的荷载
          \end{itemize}
    \item 风荷载。分项系数$1.4$。
\end{itemize}
\par 设计内容包括纵向及横向水平杆件等抗弯承载力的计算、立杆的稳定性计算、连接扣件的抗滑承载力计算、立杆地基承载力计算。
\subsection{模板安装与拆除}
\par 严密、清洁、牢固,整齐、有序、避免冲击;早拆模板体系。
\section{混凝土工程}
\subsection{混凝土的制备}
\par 配制强度和配合比。
\par 搅拌制度(搅拌时间、投料顺序)。
\par 外加剂。减水剂、引气剂、早强剂、缓凝剂、速凝剂。
\par 掺合料。粉煤灰、硅灰、粒化高炉矿渣粉。
\par 预拌商品混凝土。
\subsection{混凝土的运输}
\par 基本要求为不产生离析(否则浇筑前二次搅拌)、保证浇筑时坍落度和初凝前有充分时间进行浇筑和捣实。
\par 地面水平运输、垂直运输、高空水平运输。
\subsection{混凝土的浇筑和养护}
\par 防止离析,正确留置施工缝。
\par 大体积混凝土结构浇筑。
\par 水下浇筑混凝土用导管法。
\par 混凝土养护,人工养护和自然养护;质量检查。
\par 冬期施工。
\begin{itemize}
    \item 养护期间不加热。
          \begin{itemize}
              \item 蓄热法。
              \item 掺化学外加剂法。
          \end{itemize}
    \item 养护期间加热。
          \begin{itemize}
              \item 电极加热法。
              \item 电器加热法。
              \item 感应加热法。
              \item 蒸汽加热法。
              \item 暖棚法。
          \end{itemize}
    \item 综合。
\end{itemize}
\par 水泥不允许加热。
\chapter{预应力混凝土工程}
\par 高强钢筋、钢绞线、无粘结预应力筋。
\chapter{砌筑工程}
\section{砌筑材料}
\section{砌筑施工工艺}
\subsection{砌筑施工}
\par 抄平、放线、摆砖样、立皮数杆、挂准线、铺线、砌砖。
\par \uline{横平竖直、砂浆饱满、灰缝均匀、上下错缝、内外搭砌、接槎牢固}。
\subsection{砌石施工}
\par 毛石砌石和料石砌石。
\par 内外搭砌、上下错缝、拉结石、丁砌石交错设置。
\subsection{中小型砌块的施工}
\section{砌体的冬日施工}
\par 暖棚法、冻结法、掺盐砂浆法。
\chapter{钢结构工程}
\section{概述}
\subsection{钢结构加工流程}
\par 一般在工厂加工制作,然后运至工地进行结构安装。
\subsection{钢结构前期准备}
\section{钢结构加工工艺}
\subsection{放料、号料与下料}
\par 放样是将实行画在放样台或平板上。
\par 号料是根据样板在钢材上画出构件实样。
\par 切割下料常用方法有气割、机械切割、等离子切割。
\subsection{构件加工}
\begin{itemize}
    \item 成型加工。
          \begin{itemize}
              \item 板材卷曲。包括预弯、对中和卷曲。
              \item 管材弯曲。分为形弯、压弯和中频弯。
              \item 型材弯曲。
          \end{itemize}
    \item 边缘加工。
    \item 其他加工工艺。折边、模具压制(冲裁模、弯曲模、拉伸模和压延模)、制孔(钻孔、冲孔)。
\end{itemize}
\subsection{组装}
\par 地样法、仿形复制装配法、胎膜装配法、立装法、卧装法等。
\subsection{矫正}
\par 形式有矫直、矫平、矫形。按外力来源分为火焰矫正、机械矫正和手工矫正,按温度分为热矫正和冷矫正。
\subsection{钢结构预拼装}
\par 构件预拼装、桁架预拼装、部分结构预拼装和整体结构预拼装。方法一般分为平面预拼装(卧拼)和立体预拼装(立拼)。
\subsection{钢结构涂装、包装与运输}
\par 防腐涂装和防火涂装。
\par 涂装前表面处理。
\par 包装、运输。
\section{钢结构连接}
\subsection{钢结构焊接施工}
\par 手工电弧焊、埋弧焊、熔化极气体保护焊、电渣焊、螺柱焊(栓钉焊)。
\par 质量验收。
\subsection{钢结构螺栓连接施工}
\par 普通螺栓连接和高强度螺栓连接。
\chapter{脚手架工程}
\chapter{结构吊装工程}
\section{起重机具}
\subsection{索具设备}
\par 卷扬机、钢丝绳、锚碇。
\subsection{起重机械}
\par 桅杆式起重机(独角把杆、人字把杆、悬臂把杆和牵线式桅杆)、自行式起重机(履带式、轮胎式)、塔式起重机(轨道式、爬升式、附着式、平头式)及浮吊、缆索起重机等。
\section{构建吊装工艺}
\subsection{预制构件的制作、运输和堆放}
\par 屋架扶直。
\subsection{构件的绑扎和吊升}
\par 中小型柱绑扎一点。减少柱的吊装弯矩。
\par 斜吊绑扎法和直吊绑扎法。
\par 旋转法(边起钩,边起吊,使柱身绕柱脚旋转而逐渐吊起)和滑行法(不旋转,只起升吊钩,使柱脚在吊钩上升过程中沿着地面逐渐向吊钩位置滑行,直到柱身直立)。
\subsection{构件的就位和临时加固}
\subsection{构件的校正和最后固定}
\subsection{小型构件的吊装}
\subsection{特殊构件的吊装}
\section{大型结构安装方法}
\subsection{高空散装法}
\subsection{分块(段)吊装法}
\subsection{整体安装法}
\par 整体吊装法、整体提升法、整体顶升法、滑移安装法。折叠展开法、提升悬挑安装法、整体起扳法。
\chapter{防水工程}
\section{地下防水工程}
\subsection{结构自防水}
\par 普通结构自防水混凝土、外加剂结构自防水混凝土。
\subsection{表面防水层防水}
\par 刚性和柔性。
\par 卷材防水层施工有外贴法、内贴法。
\subsection{止水带防水}
\par 外贴式、可卸式、中埋式。
\section{屋面防水工程}
\par 平屋面和坡屋面。
\par 卷材(柔性)防水层屋面、瓦屋面、构件自防水屋面、现浇钢筋混凝土(刚性)防水屋面。
\par 卷材先高后低、先远后近。
\subsection{高分子卷材防水屋面的施工}
\subsection{涂膜防水屋面}
\chapter{装饰装修工程}
\chapter{流水施工原理}
\section{土木工程施工组织方式}
\par 施工组织活动形式一般有依次施工、平行施工和流水施工等方式。

\begin{tblr}[caption={三种施工组织方式的特点比较}]{colspec={X[c,m]X[c,m]X[c,m]X[c,m]}}
    \toprule
    比较内容           & 依次施工      & 平行施工          & 流水施工            \\
    \midrule
    工作面利用情况        & 不能充分利用工作面 & 最充分利用工作面      & 合理、充分地利用工作面     \\
    工期             & 最长        & 最短            & 适中              \\
    窝工情况           & 有窝工现象     & 若不进行协调,则有窝工现象 & 主导施工过程班组不会有窝工现象 \\
    资源用量           & 小         & 大             & 适中              \\
    资源品种           & 单一        & 单一            &                 \\
    单种资源用量均匀程度     & 不均匀       & 不均匀           & 比较均匀            \\
    对劳动生产率和工程质量的影响 & 不利        & 不利            & 有利
    \bottomrule
\end{tblr}

\subsection{依次施工}
\par 将拟建工程项目分成各个建造单元,再将各建造单元的建造过程分解为若干个施工过程,依次逐一按照一定施工顺序完成各建造单元各施工过程。
\subsection{平行施工}
\par 将拟建工程项目分成各个建造单元\footnote{通常假定各单元完全相同。},各单元在同一时间平行安排施工完成各施工过程。
\subsection{流水施工}
\par 将工程项目的全部建造过程,在工艺上分解为若干个施工过程,按横向划分为若干个施工段,按竖向划分为若干个施工层\footnote{竖向和横向不一定是空间上的竖向和横向。各施工层的施工一般有严格的顺序,而各施工段的施工的顺序大多是人为规定的。可以将施工层理解为施工段的集合,而不必拘泥于施工层与施工段的集合,并且各施工层间通常比各施工段间相似性更高。};然后按规定的施工顺序依次、连续地投入到各施工层,并使前后相邻的两个专业工作队尽可能形成无间歇搭接施工,在规定的时间内完成施工项目。
\section{流水施工参数}
\subsection{工艺参数}
\par 施工过程数$n$。
\par 流水强度$V$:每一施工过程在单位时间内所完成的工程量。\footnote{不同施工过程工程量的度量也可能不同,可参见《工程造价》。}
\par 施工过程可分为三类\footnote{比较完整的施工过程可以描述为;在制备区域制备材料或构件后运输到建造区域进行施工。可以依此理解。}:制备类施工过程,运输类施工过程,建造类施工过程。
\par 流水强度又称流水能力或生产能力。为各班生产能力之和。
\subsection{时间参数}
\par 流水节拍$K$,在一个施工段上完成一个施工过程所需的持续时间。计算方法有:定额计算法、三时估算法、工期倒排法。
\par 流水步距$B$。同一施工段上两个相邻的施工过程进入流水施工的时间差,每两个相邻施工过程间都存在一个流水步距参数。
\par 间歇时间$Z$。工艺间歇时间$\script{Z}[]{1}$,在流水步距外,相邻施工过程间施工过程工艺性质导致的间歇时间;组织间歇时间$\script{Z}[]{2}$,在流水步距外,由于组织因素要求,在相邻施工过程间增加的间歇时间。
\par 工艺搭接时间$\script{Z}[]{3}$,相邻两个施工过程的专业队在同一施工段上搭接时共同作业的时间。
\par 流水施工工期$T$,从第一个专业工作队投入流水施工到最后一个专业工作队完成流水施工的整个持续时间。
\subsection{空间参数}
\par 工作面$A$,施工对象上可能安置工人或机械的空间大小。
\par 施工段数$m$,施工段的数量。
\par 施工层$j$。
\section{流水施工的组织}
\subsection{流水施工的实施步骤}
\subsection{流水施工的表达方式\footnote{仅包含教材中最常用的表达方式。}}
\par 水平图表(横道图)。横坐标表示流水施工的持续时间,纵坐标表示施工过程、开展该施工过程的专业工作队名称或编号(从上到下,施工过程的编号需用大写罗马数字,施工队的编号用阿拉伯数字);用水平线段表示施工过程的开展顺序,同一工作队先后多次施工时应上下错开,并在每条水平线段上标注用圆圈包围的施工段编号。
\par 垂直图表(斜线图)。横坐标表示流水施工的持续时间,纵坐标表示施工段(从下到上),用斜向线段表示一个施工段上一个施工过程开始和结束施工的时间,并在斜向线段上标注施工队名称或编号。
\subsection{流水施工分类}
\par 分项、分部、单位、群体;有节奏、非节奏。
\section{有节奏流水施工}
\subsection{固定节拍施工}
\par 垂直图表为$m$条直线。
\subsection{成倍节拍施工}
\par 垂直图表为平行折线。
\section{非节奏流水}
\par 正确计算流水步距。累加数列、(相邻施工过程)错位相减、取大差法。
\section{流水施工工期的控制原理}
\chapter{网络计划技术}
\section{双代号网络图}
\subsection{基本概念}
\subsection{网络图的绘制}
\par 工艺顺序、组织顺序。
\subsection{网络图的时间参数计算}
\par 参数是写在边上。
\begin{table}[htbp]
    \centering
    \begin{tabular}{c|c|c}
        ES & LS & TF \\
        \hline
        EF & LF & FF
    \end{tabular}
\end{table}
\par 总时差$TF=LS-ES=LF-EF$,自由时差$FF=ES(post)-EF(pre)$。
\par 节点计算法。先从前到后算最早时间,再从后到前算最迟时间,接着时差。
\par 负时差绝对值最大的是关键工作。
\section{单代号网络图}
\par 工作代号、工作名称、持续时间。
\section{网络计划的优化}
\par 工期、费用、资源。
\appendix
\chapter{习题}
\begin{questionList}
    \item \begin{enumerate}
        \setcounter{enumii}{5}
        \item \begin{enumerate}
                  \item $\script{K}{a}=\tan^2(\frac{\SI{90}{\degree}-\varphi}{2})=\frac{1}{3}$,$\script{K}{p}=\tan^2(\frac{\SI{90}{\degree}+\varphi}{2})=3$。
                        \par $\gamma=\SI{18}{\kN/\m^3}$,取$\script{\gamma}{sat}=\SI{21}{\kN/\m^3}$。
                        \par $h=-0.700-(-6.200)=\SI{5.5}{\m}$,$\script{h}{T1}=h-\SI{1000}{\mm}=\SI{4.5}{\m}$。水位深度$\script{h}{w}=\SI{0.8}{\m}$。
                        \begin{align*}
                            \script{\gamma}{sat} \script{K}{a} \script{h}{c1} & = \gamma \script{K}{p} \script{h}{w}+\script{\gamma}{sat} \script{K}{p} (h+\script{h}{c1}-\script{h}{w})                                              \\
                            \script{h}{c1}                                          & = \frac{\gamma\script{K}{p}\script{h}{w}+\script{\gamma}{sat}\script{K}{p}(h-\script{h}{w})}{\script{\gamma}{sat}(\script{K}{a}-\script{K}{p})} \\
                                                                                       & =\SI{0.5482}{\m}
                        \end{align*}
                        \par 对于反弯点上部
                        \begin{align*}
                             & \script{R}{c} (\script{h}{c1}+\script{h}{T1}) + \frac{\gamma\script{K}{p}\script{h}[2]{c1}}{2} (\script{h}{T1}+\frac{2\script{h}{c1}}{3})
                            = \frac{\gamma \script{K}{a} (h+\script{h}{c1})^2}{2} (\script{h}{T1}+\script{h}{c1}-\frac{\script{h}{T1}+\script{h}{c1}}{3})                                                                \\
                             & + \frac{(\script{\gamma}{sat}-\gamma) \script{K}{a} (h+\script{h}{c1}-\script{h}[2]{w})^2}{2} (\script{h}{T1}+\script{h}{c1}-\frac{\script{h}{T1}+\script{h}{c1}-\script{h}{w}}{3})
                        \end{align*}
                        于是$\script{R}{c}=\SI{75.25}{\kN}$,$\script{h}{0}=\sqrt{\frac{6\script{R}[\prime]{c}}{\script{\gamma}{sat}(\script{K}{p}-\script{K}{a})}}=\SI{2.839}{\m}$,$\script{h}{d}=\script{h}{c1}+\script{h}{0}=\SI{3.955}{\m}$。
                  \item
                        \begin{align*}
                            \script{L}{1}+\script{L}{2} & = (h+\script{h}{c1})\tan(\frac{\SI{90}{\degree}-\varphi}{2}) + \script{h}{1}\tan(\frac{\SI{90}{\degree}+\varphi}{2}) \\
                                                              & = (5.5+0.5482)*\frac{1}{\sqrt{3}} + 1.8*\sqrt{3}                                                                           \\
                                                              & =        \SI{6.610}{\m}                                                                                                    \\
                            h \tan(\SI{90}{\degree}-\varphi)  & =    \SI{9.526}{\m}
                        \end{align*}
                        取较大值\SI{9.526}{\m}。
              \end{enumerate}
            \item   
    \end{enumerate}
\end{questionList}
\end{document}