\documentclass{book}
\usepackage{amsmath}
\usepackage[UTF8]{ctex}
\usepackage{gensymb}
\usepackage{geometry}
\usepackage{lmodern}
\usepackage{multirow}
\usepackage{textcomp}
\usepackage{tikz}

\geometry{scale=0.7}
\usetikzlibrary{positioning, shapes.geometric}

\begin{document}
\frontmatter
\title{土木工程施工基本原理提纲}
\author{H.}
\maketitle
\tableofcontents
\chapter{说明}
\par根据《土木工程施工基本原理》第2版整理。脚注表明非必要的补充说明。
\par本提纲作概览和查漏补缺用,复习时宜以考点为准。
\mainmatter
\chapter{土方工程}
\section{概述}
\subsection{土的工程分类}
\par八类土,按开挖难易程度分类,劳动定额依据。
\begin{table}[htbp]
    \caption{土的工程分类}
    \begin{center}
        \begin{tabular}{cccccccc}
            松软土 & 普通土 & 坚土 & 砾砂坚土 & 软石 & 次坚石 & 坚石 & 特坚石
        \end{tabular}
    \end{center}
\end{table}
\subsection{土的工程性质}
\par最初可松性(开挖后松散状态)和最后可松性(回填压实状态)。
\par原状土经机械压实后的沉陷,一般在$3\sim30cm$之间。
\par渗透性、密实性、抗剪强度、土压力。
\par超孔隙水压力是外荷载引起的孔隙水压力,会随着时间逐渐消散,有效应力增加,土体压缩变形,这个过程就是固结。
\section{场地设计标高的确定}
\subsection{场地设计标高确定的一般方法}
\par对于平缓且无特殊要求,场地划分为若干共$n$个方格,将角点的原地形标高(等高线插入法或实地测量)标在图上。
\par对于角点$i$,若其被$P_i$个角点共用,标高为$z_i$,则
$$
    z_0  = \frac{1}{4n} \sum^i P_i z_i
$$
\par根据泄水要求得到泄水坡度$i_x$及$i_y$,则任意点设计标高
$$
    z_i^{\prime} = z_0 + l_xi_x + l_yi_y
$$
\par各角点施工高度
$$
    H_i = z_i^{\prime} - z_i
$$
\subsection{用最小二乘法原理求最佳设计平面}
\par即最小土方量。地形复杂时将场地划分成多个,分别计算最佳设计单平面再修正。土方施工高度平方和为
\begin{align*}
    \sigma & = \sum^n P_n H_n^2                  \\
           & = \sum^n P_n(c+x_ni_x+y_ni_y-z_n)^2
\end{align*}
\par得(方括号指对表达式在所有角点上求和)
\begin{align*}
    [P]c + [Px]i_x + [Py]i_y - [Pz]     & = 0 \\
    [Px]c + [Pxx]i_x + [Pyx]i_y - [Pzx] & = 0 \\
    [Py]c + [Pxy]i_x + [Pyy]i_y - [Pzy] & = 0
\end{align*}
\par更精简一点
\begin{align*}
    [PQ]c + [PQx]i_x + [PQy]i_y - [PQz] = 0                        & ,\ Q \in \{1, x, y\}       \\
    \sum^i (P_iQ_ic + P_iQ)ix_ii_x + P_iQ_iy_ii_y - P_iQ_iz_i) = 0 & ,\ Q_i \in \{1, x_i, y_i\}
\end{align*}
\subsection{设计标高的调整}
\par最终可松性,提高设计标高。
\par工程余土或工程用土,提高或降低设计标高。
\par如采用场外取土或弃土,则应相应提高或降低设计标高。
\section{土方工程量的计算和调配}
\subsection{土方工程量计算}
\par基坑、基槽和路堤可根据其形状划分成若干拟柱,进而计算体积,$F_0$、$F_1$和$F_2$分别是中、上、下截面的面积。
$$
    V = \frac{H}{6}(F_1+4F_0+F_2)
$$
\par这是因为
\begin{align*}
    \int_a^b x^2 & = \frac{b^3-a^3}{3}                          \\
                 & = \frac{b-a}{6} (2b^2+2ab+2a^2)              \\
                 & = \frac{b-a}{6} [b^2+4(\frac{a+b}{2})^2+a^2]
\end{align*}
\par场地设计标高确定后,先通过在分别为一挖一填的两个相邻角点的连线上\footnote{插入法,根据两点分别的施工高度绘制一对三角形。}确定零点,再连接得到零线,即挖方区和填方区的交线;接着进行土方量的计算。
\par四方棱柱体法
$$
    V_p = \frac{a^2}{4} \frac{(\sum H_p)^2}{\sum H}
$$
\par三角楔柱体法需先把方格网沿地形等高线(减小绝对值)划分为三角形。公式略去。
\subsection{土方调配}
\par目的是运输量最小或成本最低,常用表上作业法。
\par划分调配区时注意与工程结构的平面位置、开工顺序、施工顺序相协调,大小满足土方施工主导机械的技术要求,范围和计算土方量时的方格网协调。土方运距较大或场地内土方不平衡时,可根据地形,就近取土或弃土,并将区域加入到调配区中。
\par调配区的平均运距和施工单价视运土工具而定,铲运机或推土机平土时平均运距为土方重心间距离。
\par运用表上作业法\footnote{详见最小费用加权流。}时,
\begin{enumerate}
    \item 挖填区分别列入表中行列,表中方格包含土方量、价格系数、假想价格系数三个量\footnote{行、列土方量之和的最大值为对应调配区的土方量。}。
    \item 先构造初始调配方案(一般使用通过最小元素法),即循环多次,每次将对应价格系数最小的土方量设为当前仍容许的最大值,直至完成调配要求。
    \item 有调配土方的方格的假想价格系数即价格系数,无调配土方的方格的假想价格系数根据任意矩形\footnote{沿水平、竖直方向。}的四个方格内对角线上假想价格系数之和相等\footnote{所以假想价格系数其实是当前方案下通过调整从别的方格调配出入当前方格的价格系数。}。即
          $$
              c_{ef}^{\prime} + c_{pq}^{\prime} = c_{eq}^{\prime} + c_{pf}^{\prime}
          $$
          校验数$\lambda_{ij} = c_{ij} - c_{ij}^{\prime}$。若校验数均为正,则方案最优,否则选择一个校验数为负\footnote{表明从其它方格调整来当前方块优于当前方案}的方格,构造包含它的仅在有调配数字的方格转弯的闭回路\footnote{同样,只有水平或竖直方向的边。},然后取奇数次转角上的调配量中最小的数字作为调整量,然后对奇数次转角上的调配量减去该数字,偶数次转角上的调配量加上该数字。重复校验调整步骤直至最优。
\end{enumerate}
\section{土方工程的准备与辅助工作}
\subsection{土方工程施工前的准备工作}
\begin{enumerate}
    \item 清理场地内地面及地下各种障碍物,包括旧房、古墓、电讯设备、管线、建筑、树木、耕植土和淤泥。
    \item 排除地面水和雨水。
    \item 修筑道路和水电等临时设施。
    \item 材料、机具和土方机械的进场。
    \item 土方工程测量放线。
    \item 根据设计做好辅助工作,如边坡稳定、基坑基槽支护、降低地下水等。
\end{enumerate}
\subsection{土方边坡及其稳定}
\par土方边坡坡度为高宽比,其倒数坡度系数为宽高比。应考虑土质、开挖深度、施工工期、地下水水位、坡顶荷载及气候条件因素。如地下水水位低于基底且开挖土层湿度正常,短期内可挖成直壁不加支撑。
\par引起下滑力增加的因素有
\begin{enumerate}
    \item 坡顶上堆物、行车等荷载;
    \item 雨水或地面水渗入土中使含水量提高,自重增加;
    \item 地下水渗流产生动水压力;
    \item 土体竖向裂缝中的积水产生侧向静水压力等。
\end{enumerate}
\par引起抗剪强度降低的因素有
\begin{enumerate}
    \item 气候使土质松软;
    \item 土体含水量增加产生润滑;
    \item 饱和的细沙、粉沙受振动而液化。
\end{enumerate}
\par因此,要注意及时排除雨水、地面水,防止坡顶集中荷载及振动,必要时可采用钢丝网细石混凝土(或砂浆)护坡面层加固。永久性土方边坡则应做好永久性措施。
\subsection{土壁支护}
\par建筑稠密或有地下水渗入时多不可能按要求的坡度放坡开挖,这就需要基坑基槽支护。其主要作用是支撑土壁,钢板桩、混凝土板桩和水泥土搅拌桩等还兼有不同程度隔水作用。根据受力状态可分为
\begin{enumerate}
    \item 横撑式支撑\footnote{顾名思义,两边挡土板通过其间的横撑对撑。}。
          开挖较窄的沟槽时多用。
          \begin{itemize}
              \item 水平挡土板式\footnote{横撑固定着的立柱支撑着横板作用于土壁上。}。
                    \begin{itemize}
                        \item 间断式。湿度小的黏性土,挖土深度小于$3m$。
                        \item 连续式。松散、湿度大的土。
                    \end{itemize}
              \item 垂直挡土板式\footnote{横撑支撑着竖板作用于土壁上。}。松散和湿度很高的土,挖土深度不限。
          \end{itemize}
          土压力的分布于土的性质、土坡高度、支撑的形式和变形有关,通常分密砂、松砂和黏土几种简化图形进行计算。
    \item 土钉墙\footnote{在土体中设钢筋或钢管(土钉,一般斜向下)并结合坡面的钢筋网喷射混凝土面板。}。主动嵌固作用。土钉群与土体共同作用,制约土体变形,并与之构成整体(复合土体),大大提高原土体强度和刚度,还能改变变形与破坏形态。\footnote{可以设想土体有沿滑动面滑动的趋势时,土钉群和面板的反作用力。}可以与止水帷幕、微型桩等复合。
          \par适用于地下水位上或降水后的人工填土、粘性土和弱胶结沙土,深度不大于$5m$,不宜用于含水量丰富的细沙、淤泥质土和砂砾卵石层以及没有自稳能力的淤泥和饱和软弱土层。\footnote{土钉墙需要土钉与土体之间有一定黏结强度,并且是让复合土体共同承担自重和荷载而没有外加支撑。}
          \par计算土钉参数时,假定破裂面是与水平面成$\frac{\beta+\varphi_{ak}}{2}$的平面\footnote{从坡脚出发向土体内延伸。取平面应为简化故。}。假定破裂面外的土体分层计算。
          \begin{itemize}
              \item 计算土钉长度时按照假定破裂面外总黏结强度不小于轴向荷载标准值与安全系数$1.4$之积的原理计算,轴向荷载标准值与土钉倾角的余弦值之积为相应范围内土体的侧压力与坡面倾斜的荷载折减系数之积。
              \item 计算土钉杆体截面面积时按照杆体截面抗拉强度不小于土钉验收抗拔力与安全系数$1.15$之积的原理计算,验收抗拔力为假定破裂面外总黏结强度与土钉的工作系数(取$0.8\sim1.0$)。
          \end{itemize}
          \par整体稳定性分析可采用简化圆弧滑移面条分法。按照土、土钉、预应力锚杆、截水帷幕及微型桩\footnote{通常来说抗滑作用由大到小。}各自的整体稳定性分项抗力系数(产生的抗滑力矩与土体下滑力矩之比)与相应的组合作用折减系数(土没有该系数,乘以1即可)之积的和不小于整体稳定性安全系数的原理计算。
          \par还应验算坑底抗隆起稳定性,有透水可能的还应验算抗渗流稳定性,坑底存在承压水时还应验算抗突涌稳定性。
          \begin{figure*}[htbp]
              \caption{土钉墙施工工艺流程}
              \begin{center}
                  \begin{tikzpicture}[node distance=10pt]
                      \node[draw] (step 1)  {开挖工作面};
                      \node[draw, right=of step 1] (step 2)  {修正边坡};
                      \node[draw, right=of step 2] (step 3)  {喷射第一层混凝土};
                      \node[draw, right=of step 3] (step 4)  {钻孔};
                      \node[draw, right=of step 4] (step 5)  {安设土钉};
                      \node[draw, below=of step 5] (step 6)  {注浆};
                      \node[draw, left =of step 6] (step 7)  {绑扎钢筋网};
                      \node[draw, left =of step 7] (step 8)  {喷射第二层混凝土};
                      \node[draw, left =of step 8] (step 9)  {下一个工作面施工};

                      \draw[->] (step 1) -- (step 2);
                      \draw[->] (step 2) -- (step 3);
                      \draw[->] (step 3) -- (step 4);
                      \draw[->] (step 4) -- (step 5);
                      \draw[->] (step 5) -- (step 6);
                      \draw[->] (step 6) -- (step 7);
                      \draw[->] (step 7) -- (step 8);
                      \draw[->] (step 8) -- (step 9);
                  \end{tikzpicture}
              \end{center}
              \caption{复合土钉墙施工工艺流程}
              \begin{center}
                  \begin{tikzpicture}[node distance=10pt]
                      \node[draw] (step 1)  {止水帷幕施工};
                      \node[draw, right=of step 1] (step 2)  {微型桩施工};
                      \node[draw, right=of step 2] (step 3)  {开挖工作面};
                      \node[draw, right=of step 3] (step 4)  {修整边坡};
                      \node[draw, below=of step 4] (step 5)  {土钉施工};
                      \node[draw, left =of step 5] (step 6)  {挂网、喷射混凝土施工};
                      \node[draw, left =of step 6] (step 7)  {锚杆施工};
                      \node[draw, left =of step 7] (step 8)  {下一个工作面施工};

                      \draw[->] (step 1) -- (step 2);
                      \draw[->] (step 2) -- (step 3);
                      \draw[->] (step 3) -- (step 4);
                      \draw[->] (step 4) -- (step 5);
                      \draw[->] (step 5) -- (step 6);
                      \draw[->] (step 6) -- (step 7);
                      \draw[->] (step 7) -- (step 8);
                  \end{tikzpicture}
              \end{center}
          \end{figure*}
          \par自上而下分段分层(每层高度取决于土体直立能力,一般取土钉竖向间距以便施工)进行,机械开挖后辅以人工修正。
          \par喷射第一层混凝土支护前,坡面需干燥平实,喷射时分段,段内自下而上,厚度不小于$40mm$,喷头保持与受喷面垂直,距离为$0.6\sim1m$,而后铺设钢筋网,间隙不小于$20mm$。
          \par土钉安设方法有钻孔插入法(需全长注浆,可分一次到两次)和直接打入法,上层土钉注浆体(若有注浆)及喷射混凝土面层达到设计强度的$70\%$后方可开挖下层土方。
          \par施工完毕后进行土钉抗拉、喷射混凝土钻孔检测。
    \item 板桩式支护结构,又分为悬臂式和支护式。支护式由挡墙系统和支撑或拉锚系统(悬臂式不设)两大系统组成。悬臂式位移较大,多用于$3\sim4m$深的浅基坑,一般广泛采用支撑式板桩。
          \par挡墙系统常用槽钢、钢板桩(平板式和波浪式\footnote{隔水能力强,波浪式截面抗弯能力好。},通过锁口互相连接,完毕后可拔出重复使用)、钢筋混凝土板桩、灌注桩、地下连续墙和$SMW$工法等。
          \par支撑系统一般采用大型钢管、$H$型钢、格构式钢支撑或现浇钢筋混凝土支撑。拉毛系统的材料一般用钢筋、钢索、型钢或土锚杆。根据开挖深度和挡墙系统的截面性能可设一道或多道支点。
          \par板桩工程事故主要有五方面原因
          \begin{itemize}
              \item 板桩的入土深度不够,入土部分移动而滑坡;
              \item 支撑或拉锚的强度不够;
              \item 拉锚长度不足,锚碇失去作用;
              \item 板桩刚度不够,失稳弯曲;
              \item 板桩位移过大,周边环境破坏。
          \end{itemize}
          \par因此,板桩的设计五大要素为入土深度、截面弯矩、支点反力、拉锚长度和板桩位移。
          \par以单支点\footnote{支点指支撑和拉锚。}为例,根据入土深度与基坑深度比值的大小,单支点板桩变形也不同,分为自由支承和嵌固支承。\footnote{前者入土浅,整个板桩都向坑内变形,底端转动且无位移或有微小位移,桩前土发挥被动土压力。后者底端有一段既无位移也无转角。}
          \par对于单支点嵌固板桩,常用简化计算方法---相当梁法。将土压力简化为线性分布,则板桩有反弯点\footnote{反弯点上方的部分即相当梁。},简化取反弯点在土压力零点\footnote{桩前后被动、主动土压力在该点相等。该简化可避免求支承力。由于线性,相当梁上桩前后土的合力也相等,反弯点剪力与支承力平衡。},再根据桩前后土压力求出反弯点下另一个弯矩零点。两零点之间的长度需要乘以$1.2$以修正,与反弯点入土深度相加得到板桩入土深度;在两零点之间剪力为零处\footnote{由于线性,为中点。}求得最大弯矩。
          \par拉锚长度应保证锚碇或锚座板在静止土楔滑移线之外,它引起的被动土楔滑移线和板桩位移引起的主动土楔滑移线不能相交。\footnote{主动、被动土楔滑移线与受力(主动为重力)方向成$(90\degree-\varphi)/2$角斜向外,静止土楔滑移线与受力方向成$90\degree-\varphi$角斜向外。注意计算嵌固板桩时,板桩位移引起的被动土楔滑移线以反弯点为起点。}
          \par围檩\footnote{支护板桩上部的横梁,起连接固定作用。}可用型钢或现浇混凝土结构,前者按简支梁计算,后者按连续梁计算。
          \par一般钢板桩、混凝土板桩采用打入法。灌注桩和地下连续墙采用就地成孔(槽)现浇的方法。板桩施工要正确选择打桩方法、打桩机械和流水段划分,以保证刚度和防水性,且墙面要平直,封闭式板桩墙还要封闭合拢。
          \par钢板桩的打桩方法通常有单独打入法\footnote{一角开始逐块插打,每块不停顿,打设快,但易向一边倾斜不易纠正,墙体平直度难控制,一般长度小于$10m$。}、围檩插桩法\footnote{用围檩支架作板桩打设导向装置。分单面围檩和双面围檩。可以保证尺寸准确和钢板桩垂直,但施工速度慢。}、分段复打法\footnote{一定数目钢板桩沿单层围檩插入土中从两端开始依次打入,打好后电焊固定在围檩上,最终形成屏风墙。桩锤重量一般为钢板桩2倍,不能过宽。}。
          \par地下工程结束后钢板桩一般都要拔出,需正确选取方法和顺序,并应注浆填充。
    \item 重力式支护结构。利用深层搅拌桩机将水泥与土一起搅拌,形成柱状的加固土(搅拌桩),用于支护结构时水泥掺量(单位体积掺入水泥质量与土的重力密度之比)通常为$12\%\sim15\%$,渗透系数很小。适用于$4\sim6m$的基坑,最大$7\sim8m$。
          \par设计主要包括整体稳定、抗倾覆稳定、抗滑移稳定、位移等。有时还应验算抗渗、墙体应力、地基强度等。施工质量对位移的影响不可忽略。
          \par通常布置成格栅式。置换率(加固土面积与水泥土墙总面积之比)为$0.6\sim0.8$。
          \par常用深层搅拌桩机分动力头式和转盘式两大类。(多头可在一个工艺流程施工多根桩)$SMW$工法可以利用装有三轴搅拌钻头的钻机在地层中连续建造水泥土墙并插入芯材。
          \par成桩工艺可采用“一次喷浆、二次搅拌”或“二次喷浆、三次搅拌”工艺(水泥掺量小、土质松时可用前者,反之可用后者)。应注意水泥浆配合比、搅拌制度、水泥浆喷射速率与提升速率的关系和每根桩的水泥浆喷注量,以保证注浆的均匀性与桩身强度;还应注意桩的垂直度、桩的搭接,以保证水泥土墙的整体性和抗渗性。
\end{enumerate}
\subsection{降水}
\par降水方法可分为重力降水和强制降水\footnote{用水量较大、水位较大或土质为细沙、粉沙,易产生流沙、边坡塌方及管涌等时往往采用。}。
\par集水井降水是在开挖时在坑底设置集水井,井沿周围或中央开挖排水沟,使水在重力下流入,用水泵抽出。一般设在基础范围外、地下水流上游,基坑较大时可以在基础范围内设置盲沟。根据地下水量、基坑平面形状、水泵能力,每隔$20\sim40m$设一集水井,一般宽$0.6\sim0.8m$,深度随挖土加深,通常低于挖土面$0.7\sim1.0m$,井壁可用竹、木等简单加固。挖至设计标高时井底应低于坑底$1\sim2m$,并设碎石滤水层。集水井降水比较简单经济,影响小。
\par井点降水是在开挖前预先埋设滤水管(井),利用真空原理不断抽出地下水。作用有
\begin{itemize}
    \item 防止地下水涌入坑底,防止边坡由于地下水的渗流引起塌方,消除地下水位差导致的坑底土层压力,防止坑底的管涌;
    \item 减少板桩的横向荷载,消除地下水渗流,防止出现流沙;
    \item 使土壤固结,增加地基的承载能力。
\end{itemize}
\par易知水对土颗粒的作用方向与渗流一致,大小为水力坡度和浮重度之积。防治流沙的方法主要有水下挖土法、冻结法、枯水期施工、抢挖法、加设支护结构和降水法等。
\par井点降水法有轻型井点和管井两类。轻型经典设备由管路系统\footnote{从末到本为滤管、井点管、弯联管、短接头和总管等。}和抽水设备\footnote{真空泵、离心泵和水气分离器(又称集水箱)。}组成。井点系统布置应根据水文地质资料、工程要求和设备条件等确定。
\par轻型井点布置包括平面布置(井点布置形式、总管长度、井点管数量、水泵数量及位置等)和高程布置(埋设深度)。布置和计算时先确定平面布置,再确定高程布置(确定滤管上口至总管埋设面距离),接着计算井点管数量等,最后根据问题调整设计。平面布置中
\begin{itemize}
    \item 单排布置适用于基坑(槽)宽度小于$6m$且降水深度不超过$5m$的情况。应布置在地下水上游侧,两侧延伸长度不宜小于坑(槽)宽度。
    \item 双排布置适用于基坑宽度大于$6m$或土质不良的情况。
    \item 环形布置适用于大面积基坑。
    \item $U$形布置,不封闭的一段应设在地下水的下游。
\end{itemize}
\par井点管布置应离坑边$0.7\sim1m$,以防塌土漏气。
\par一般基底至降低后的地下水位线距离取$0.5\sim1m$,降水后的水力坡度(假设恒定)为
\begin{itemize}
    \item 单排布置时$1/4\sim1/5$;
    \item 双排布置时$1/7$;
    \item 环形布置时$1/10$。
\end{itemize}
\par实际工程中,井点管定型,通常露出地面$0.2m$。总管长度根据基坑上口尺寸或基槽长度确定,进而根据水泵负荷长度确定水泵数量。
\par为了确定井点管数量还需计算井点系统涌水量。水井根据地下水有无压力,分为承压井和无压井,根据底部是否为不透水层,分为完整井和非完整井。
\par计算无压完整井时,假定某一过水面(以井点为中心的圆柱面)水力坡度是恒值,根据含水层厚度、断面流量恒定和达西定律可求出水位降落曲线和流量。非完整井的含水层厚度取有效含水深度,有效含水深度根据井点管处水位降落值和井点管中水位计算查表得到,但不超过实际含水层厚度。
\par轻型井点施工过程分为准备工作(井点设备、动力、水源、必要材料的准备、排水沟的开挖、附近建筑物的标高观测和实施防止附近建筑物沉降的措施)、井点系统的埋设(先设置总管,再设井点管,用弯联管将井点与总管接通,然后安装抽水设备。一般用水冲法进行,分为冲孔与埋管两个过程。需注意不让滤管触底、保证砂滤层填灌质量,并用粘土封口。开始抽水后一般不应停抽)、使用和拆除。
\par轻型井点降水优点许多,但抽水影响范围较大,可达百米至数百米,且会导致周围土壤固结、地面沉陷。可采用回灌井点法消除地面沉陷,将抽取的水沉淀后注入井点外数米的管内形成水墙,此时抽水管的抽水量约增加$10\%$,可适当增加抽水井点的数量。
\par管井井点一般有疏干井(还需降低土体含水量以达到提高边坡稳定性、增加坑内土体固结强度、便于机械挖土和提供坑内干作业施工条件等目的时)、降压井(软土地区承压水一般接近上覆土层自重应力,此时基坑开挖后需降低承压水压力以防止基坑突涌)和混合井三种。
\subsection{基坑支护工程的现场监测}
\par由于地下工程影响因素比较复杂,实际工作状况往往与设计计算值不一致,需对基坑支护工程现场检测。设计阶段,设计人员根据具体情况,对现场检测提出观测项目、测点布置、观测精度、观测频度和临界状态报警值等具体要求,监测人员据其在基坑开挖前制定出包括检测目的与内容、测点布置、使用的仪器、检测精度、观测方法、观测周期、监测项目报警值、检测结果处理要求和监测结构反馈制度等现场检测方案。
\par支护结构检测主要包括(标下划线的项目不可缺少)
\begin{itemize}
    \item \underline{水平位移监测}、竖向位移监测、深层水平位移监测、\underline{沉降监测}、倾斜检测;
    \item \underline{裂缝检测};
    \item 支护结构内力监测、土压力监测、孔隙水压力监测;
    \item 地下水位监测。
\end{itemize}
\par基坑工程监测方法应根据基坑类别、设计要求、场地条件、当地经验和方法适用性等因素综合确定,以仪器(根据检测项目选择,主要有水准仪、全站仪、测斜仪、回弹仪、轴力仪、土压力计和水压力计等,需满足观测精度、量程、稳定性、可靠性的要求)观测为主,辅以巡视检查。
\par监测的基本要求有
\begin{itemize}
    \item 严格执行;
    \item 动态及时观测;
    \item 可靠,精度符合要求;
    \item 发现异常及时预报、应急补救;
    \item 观测记录完整,有形象的图表、曲线和观测报告。
    \item 监控值以设计指标或规定为依据。
\end{itemize}
\section{土方工程的机械化施工}
\par施工过程包括土方开挖、运输、填筑和压实。尽量采用机械化施工。主要挖土机械有
\begin{itemize}
    \item 推土机。以切土和推运土方为主,适于开挖一至三类土,多用于平整场地、开挖深度不大的基坑,移挖作填,回填土方,堆筑堤坝以及配合挖土机集中土方、修路开道等。\footnote{履带式拖拉机上安装推土板等而成,多用油压操纵。操纵灵活,运转方便,所需工作面较小,行驶速度快,易于转移,能爬$30\degree$左右的缓坡,应用范围较广。经济运距在$100m$以内,效率最高为$60m$。可采用槽型推土、下坡推土以及并列推土等方法提高效率。}
    \item 铲运机。适于开挖一至三类土,常用于坡度为$20\degree$以内的大面积土方挖、填、平整、压实,大型基坑开挖和堤坝填筑等。\footnote{按行走方式分为自行式和拖式两种,按铲斗操纵系统可分为机械操纵和液压操纵。操纵简单,不受地形限制,能独立工作,行驶速度快,生产效率高。线路可采用环形路线或$8$字路线。适用运距为$600\sim1500m$,效率最高为$200\sim350m$。可采用下坡铲土、跨铲法、推土机助铲法等缩短装土时间,提高土斗装土量。}
    \item 挖掘机。\footnote{按行走方式分为履带式和轮胎式,按传动方式分为机械传动和液压传动。}
    \begin{itemize}
        \item 装载机。主要用于铲装土壤、砂土、石灰、煤炭等散状物料,也可对矿石、硬土等作轻度铲挖作业。\footnote{作业速度快、效率高、机动性好、操作轻便。}
        \item 正铲挖掘机。适于开挖含水量较小的一至四类土和经爆破的岩石及冻土,挖土时\underline{前进向上,强制切土}。\footnote{生产率主要决定于每斗作业的循环延续时间,为提高生产率,除工作面高度满足装满土斗的要求外,还要考虑开挖方式、与运土机械配合。}
        \item 反铲挖掘机。适于开挖一至三类的沙土或黏土。主要用于开挖停机面以下的土方,挖土时\underline{后退向下,强制切土}。一般最大挖土深度为$4\sim6m$,经济合理为$3\sim5m$。\footnote{需要配备运土汽车,可以采用沟端开挖法或沟侧开挖法。}
        \item 抓铲挖掘机。适用于开挖较松软的土。挖土时\underline{直上直下,自重切土}。\footnote{对施工面狭窄而深的基坑、深槽、深井采用。}还可用于挖取水中淤泥(易被吸住,谨防翻车),装卸碎石、矿渣等松散材料。
        \item 拉铲挖掘机。适用于一至三类土,可开挖停机面以下的土方,如较大基坑基槽和沟渠,挖取水下泥土,也可用于填筑路基、堤坝等,挖土时\underline{后退向下,自重切土}。但开挖的边坡及坑底平整度较差。开挖方式有沟端开挖和沟侧开挖。
    \end{itemize}
\end{itemize}
选择土方机械应依据土方工程的类型及规模\footnote{开挖或填筑的断面深度及宽度、工程范围的大小、工程量的多少}、地质水文气候条件、机械设备条件来选择
\end{document}